% ************************** Thesis Abstract *****************************
% Use `abstract' as an option in the document class to print only the titlepage and the abstract.
%\title{Graphical pangenomics}
%\author{Erik Peter Garrison}
\begin{abstract}
  %\textbf{Graphical pangenomics} \\
  %\textbf{Erik Peter Garrison} \\
  Completely sequencing genomes is expensive, and to save costs we often analyze new genomic data in the context of a reference genome.
  This approach distorts our image of the inferred genome, an effect which we describe as reference bias.
  To mitigate reference bias, I repurpose graphical models previously used in genome assembly and alignment to serve as a reference system in resequencing.
  To do so I formalize the concept of a \emph{variation graph} to link genomes to a graphical model of their mutual alignment that is capable of representing any kind of genomic variation, both small and large.
  As this model combines both sequence and variation information in one structure it serves as a natural basis for resequencing.
  By indexing the topology, sequence space, and haplotype space of these graphs and developing generalizations of sequence alignment suitable to them, I am able to use them as reference systems in the analysis of a wide array of genomic systems, from large vertebrate genomes to microbial pangenomes.
  To demonstrate the utility of this approach, I use my implementation to solve resequencing and alignment problems in the context of \emph{Homo sapiens} and \emph{Saccharomyces cerevisiae}.
  I use graph visualization techniques to explore variation graphs built from a variety of sources, including diverged human haplotypes, a gut microbiome, and a freshwater viral metagenome.
  I find that variation aware read alignment can eliminate reference bias at known variants, and this is of particular importance in the analysis of ancient DNA, where existing approaches result in significant bias towards the reference genome and concomitant distortion of population genetics results.
  I validate that the variation graph model can be applied to align RNA sequencing data to a splicing graph.
  Finally, I show that a classical pangenomic inference problem in microbiology can be solved using a resequencing approach based on variation graphs.
\end{abstract}
