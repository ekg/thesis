%!TEX root = ../thesis.tex
%\input{commands.tex}
%*******************************************************************************
%****************************** Third Chapter **********************************
%*******************************************************************************
\chapter{Conclusions}

% **************************** Define Graphics Path **************************
\ifpdf
    \graphicspath{{Chapter3/Figs/Raster/}{Chapter3/Figs/PDF/}{Chapter3/Figs/}}
\else
    \graphicspath{{Chapter3/Figs/Vector/}{Chapter3/Figs/}}
\fi

Genomics is driven by comparison.
It is rare that we have reason to consider a single genome in isolation.
We apply alignment algorithms to infer the sequence-level relationship between genomes.
As data scales have increased, we have required incomplete methods to determine these relationships among many individuals.
Contemporary resequencing techniques focus on the placement of new sequencing information into a reference system which is typically linear and representative of only a single copy of each genomic locus.
A pangenomic reference system could allow us to represent multiple versions of each, but until recently such techinques have been difficult to apply at scales commonly reached in current analyses.

I propose the use of variation graphs as reference systems in resequencing.
These path-labeled, bidirectional DNA sequence graphs allow us to represent collections of genomes in a single, coherent structure which fully capture the sequences and variation between them.
They support the direct representation of all kinds of genomic variation.
By building a software system supporting the construction, manipulation, indexing, and alignment of new read sets and genomes to variation graphs, I am able to shown that this model reduces bias towards the reference in alignment in a wide array of genomic contexts.
These methods achieve a level of performance that will make them usable for large-scale resequencing analyses.
As I have shown, their modular implementation, based around a handful of core data models, enables the rapid construction of novel graph-based analysis processes that provide conceptual unity to alignment, assembly, and variant calling.
Although many other methods for aligning sequences against pangenome data structures exist, {\tt vg} is the first set of tools that does so in a completely coherent manner against arbitrary bidirectional sequence graphs.
This is the first framework to provide graph based analogs of the data types used in resequencing.

In addition to developing methods to support alignment to variation graphs, I have explored a wide array of related analyses.
I developed new techniques for visualizing variation graphs that will help to build the genome browsers necessary to navigate data placed in the context of the graph.
I have linked the variation graph data model to RDF, which in principle will allow for high-level inference based on the genomes and their annotations embedded in a variation graph. 
I provide a method to losslessly induce variation graphs from a set of sequences and pairwise alignments between them using external memory.
I built systems that allow for the efficient summarization of alignment data sets against variation graph, and worked on methods to support genotyping known and novel variation in graphs.
Throughout my work I have supported and worked with a growing group of researchers focused on these techniques, collaborating in the development of grap sequence and haplotype indexing techinques, the evaluation of diverse variation graph models, and the study of ancient DNA using variation graphs.

Graphical models are often regarded with apprehension by members of the bioinformatics community who are accustomed to working with linear reference genomes.
I show that arbitrary variation graphs may be consistently linearized for visualization and analysis.
Variation graphs built from related sequences tend to have a manifold linear propery despite the frequent presence of large scale variation.
I show that this holds for graphs constructed from a variety of sources using alignment or assembly techinques retain relatively linear structures locally, and as such can be used for efficient alignment.
The linearization of the graph suggests a projection of sequencing information in the graph into a basis vector space defined by the graph itself.
Such an approach may greatly simplify genomic analyses by removing the complicated variant calling step.
If the variation we want to consider is already embedded in the graph, we do not need to genotype novel variation or engage in filtering of our results.
As variation is now embedded in the graph, we can perhaps avoid variant calling altogether where downstream it is possible to work with a normalized coverage model across this graph basis vector.
Doing so practically will require the development of techinques that can scale genetic analyses to the large matrix representations implied by a such maps.

It is not clear how to build the best graph for a given analysis context.
The results I present show that the addition of variation to a graph does not necessarily improve alignment performance in all contexts.
Additional variation increases graph complexity, and this can make results more ambiguous.
One important step is the use of haplotype information at the level of alignment.
Ongoing work suggests that doing so may mitigate scaling issues that will occur as we build graphs from tens and hundreds of thousands of genomes, but there is still much work to be done.
Presumably, with time, practices will arise that capture the ideal patterns for constructing variation graphs.
I found a number of potential input sources unreliable in their current form, and I hope to explore them as variation graph analysis techniques mature.
Progressive and multiple whole genome alignment algorithms have difficulty scaling to more than single human chromosomes, but they are the only reliable way to merge haplotype resolved genome assemblies that new genome inference technologies are enabling.
I was unable to use assemblers to construct non-fragmented string graphs from long-read sequencing data, which motivated me to build a variation graph inducer capable of building an input graph for use in an assembly process.
If it can be made to scale, this could stand as the first step in a reconfigurable haplotype aware assembly process based on the algorithms already implemented in {\tt vg}.
Provided improvement of the input alignment process, this technique could also serve as a scalable way to construct variation graphs in any context where collections of sequenced genomes exist.

I believe that reference genomes should be replaced with pangenomic structures.
This is the clearest way to resolve represenational issues that arise as we collect large collections of genomes in the species we examine.
The variation graph is a natural model to do so.
Its adoption is now a social question.
Can the community generate a unified set of data structures that encapsulate the ideas I have presented eher?
Large distributed projects like the 1000GP gave rise to the current generation of genomic data formats.
It seems natural that the next, graphical, pangenomic phase will require the same.
At present it is not clear what project might support this.
Top-down approaches like that presented by the GA4GH have not proven as capable of promoting standards as analysis-oriented projects like the 1000GP, although they have served as coordinating role for the community of researchers interested in these topics.
One obvious target for the widespread introduction of variation graph data models would be in the generation of a new reference genome system based on a collection of fully-resolved genomes.
Motivation for such an advance increases as evidence mounts that the mass of genetic variation is neither small nor simple.
I am hopeful that my work may support such an effort, and that the ideas which arise therein may follow from the generic graphical pangenomic models I have proposed and validated here.
