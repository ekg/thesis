%!TEX root = ../thesis.tex
%*******************************************************************************
%*********************************** First Chapter *****************************
%*******************************************************************************

\chapter{Introduction}  %Title of the First Chapter

\ifpdf
    \graphicspath{{Chapter1/Figs/Raster/}{Chapter1/Figs/PDF/}{Chapter1/Figs/}}
\else
    \graphicspath{{Chapter1/Figs/Vector/}{Chapter1/Figs/}}
\fi


\section{Genome sequencing and inference}

\subsection{DNA sequencing techniques}

% Genome sequencing techniques
% Sanger (1st gen), illumina (2nd gen), single molecule ONT/PacBio (3rd gen)

\subsection{The reference sequence}

% history of reference sequences
% use of the reference sequence in analysis

\subsection{The reference bias problem}

% it is necessarily harder to see things when the become more divergent
% literature review of examples

\subsection{Pangenomic solutions to reference bias}

% think through, with references, the various ways we can resolve reference bias
% extended 

\section{Graphical applications in sequence analysis}

\subsection{Multiple sequence alignment}

\subsection{Assembly graphs}

\subsection{deBruijn graphs}

\subsection{Overlap graphs}

\subsection{String graphs}

\subsection{RNA splicing graphs}

\subsection{Variant representation with DAGs (VCF)}

\section{Overview of this work}

% text
