%!TEX root = ../thesis.tex
%*******************************************************************************
%****************************** Second Chapter *********************************
%*******************************************************************************

\chapter{Variation graphs}

\ifpdf
    \graphicspath{{Chapter2/Figs/Raster/}{Chapter2/Figs/PDF/}{Chapter2/Figs/}}
\else
    \graphicspath{{Chapter2/Figs/Vector/}{Chapter2/Figs/}}
\fi

Graphical models used to compress collections of genomic sequence data are unable to losslessly reproduce their own input.
This is plainly seen by considering the an alignment graph where two homologous sequences have an internal homology between them.
Such a graph will represent four possible input sequences as walks through its nodes and edges, and without additional labeling it will be impossible to obtain its input from the graph itself \cite{kehr2014genome}.

While the input sequences exist in the set of sequences that can be generated by walks through the graph, as the complexity of the graph grows the number of paths will increase exponentially.
If annotated with transition weights between nodes representing the frequency which different edges are taken by sequences in the input, then the graph can be used as an HMM to simulate new sequences.
While these will have local similarity to the graph's preimage, they are unlikely to provide a good model for the input sequences.
Sequences generated from an HMM will exhibit an exponential decay in mutual information between their symbols, while natural sequences tend to exhibit power-law decay in mutual information \cite{lin2017critical}.
This observation suggests that no pure markovian model will allow us to faithfully represent its input, and in turn that the only way to be sure that we model the input sequence space is to maintain a record of it in the context of the graph.
The compression provided by the graph is still relevant, as we can exploit the repetitive nature of natural sequences to obtain compression of this set, but using lossless techniques.

\emph{Variation graphs} (VG)\footnote{I will refer to variation graph as VG, and to the software implementation of the VG model {\tt vg}}, previously introduced in section \ref{sec:the_variation_graph}, combine a bidirectional sequence graphs with paths that model their preimage as walks through the positional space of the graph.
This makes them representationally equivalent to both graphical models and linear sequence models, and allows them to be used to model the relationships between collection of sequences, including all variation contained therein.
The encapsulation of these two divergent ways of modeling about bioinformatic data systems allows them to bridge traditionally isolated analysis modalities.
We can define these graphs by their construction from a set of pairwise alignments between sequences, as in the alignment, Enredo, and Cactus graph models, but if we want to relate new sequences to them then we would need to rebuild the entire graph to include these sequences.
By developing a model for an alignment against the variation graph we can express the relationship between new sequences and the graph without the cost of embedding the new sequence in the graph.
This allows us to follow the same kinds of analysis patterns used in resequencing, where a common reference system is used to compute a form of joint assembly between (potentially) many samples in an out of core manner.
Further, in such a scheme the reference system can remain relatively stable, which has relevance for practical use.

In this chapter, I will articulate the variation graph model and lay out the algorithms and data structures that enable its use as a reference system in pangenomic resequencing.
First I will provide formulations for the graph, its paths, edits, alignments, and genotypes define within it.
Then I will present algorithms that induce the variation graph from different data models introduced in the previous chapter.
To clarify the system's practical basis, I explain the serialization techniques used to exchange variation data.
A main focus of my work has been the efficient alignment of sequences to variation graphs.
This requires index structures to store the graph and enable queries of its topology, sequence, and path spaces, and algorithms to drive the inference of optimal alignments to the graph.
Understanding variation graphs requires techniques to visualize them, and I will present various approaches, each with particular advantages and drawbacks.
Working with variation graph references necessitates a number of mutating operations to the graph, including augmentation, sorting, pruning, and bubble simplification.
Finally, I will discuss how variation graphs can provide normalized basis spaces for the analysis of pangenomes, such as through various decompositions of alignment sets and the graph including coverage maps, ultrabubble decomposition, and haplotype matching.

\section{An extensible graphical model of many sequences}

We define a variation graph to be a graph with embedded paths $G = (N, E, P)$ comprising a set of \emph{nodes} $N = n_1 \ldots n_M$, a set of \emph{edges} $E = e_1 \ldots e_L$, and a set of \emph{paths} $P = p_1 \ldots p_Q$, each of which describes the embedding of a sequence into the graph.
At its heart, this model is simply a bidirectional sequence graph with a recorded set of walks through the graph topology.
By generalizing these paths to encode edits against the graph, we provide a mechanism to describe relations between the graph and other sequences.
Augmenting the path with additional information important to sequence analysis allows us to construct an \emph{alignment}, which forms the core data type in resequencing.
Collections of pairs of paths covering the space of two graphs describe a graph to graph alignment, or \emph{translation} which can be generated when the graph is edited to allow for the projection of coordinates and sequences in one graph into the space fo the other.
A limited form of this translation is a \emph{genotype}, which maps the implied bubble formed across multiple copies of a homologus locus in one genome into the space of the graph.
Collections of genotypes are the primary output of resequencing.
Phasing algorithms extend genotypes into longer phased haplotypes, which have a natural representation as paths through the graph.
These data models thus provide a sufficient informational basis for resequencing against variation graphs.

\subsection{The bidirectional sequence graph}

Each node $n_i$ represents a sequence $seq(n_i)$ that is built from an alphabet $\Sigma = \{ {\tt A, C, G, T, N} \}$. Nodes may be traversed in either the forward or reverse direction, with the sequence being reverse-complemented in the reverse direction.
We write $\overline{n_i}$ for the reverse-complement of node $n_i$, so that $seq(n_i) = revcomp(seq(\overline{n_i}))$.
Note that $n_i = \overline{\overline{n}}$. For convenience, we refer to both $n_i$ and $\overline{n_i}$ as ``nodes''.

Edges represent adjacencies between the sequences of the nodes they connect.
Thus, the graph implicitly encodes longer sequences as the concatenation of node sequences along walks through the graph.
Edges can be identified with the ordered pairs of oriented nodes that they link, so we can write $e_{ij} = (n_i,n_j)$.
Edges also can be traversed in either the forward or the reverse direction, with the reverse traversal defined as $\overline{e_{ij}} = (\overline{n_j},\overline{n_i})$.
Note that graphs in vg can contain ordinary cycles (in which $n_i$ is reachable from $n_i$), reversing cycles (in which $n_i$ is reachable from $\overline{n_i}$), and non-cyclic instances of reversal (in which both $n_i$ and $\overline{n_i}$ are reachable from $n_j$).

\subsection{Paths with edits}

We implement paths as an edit string with respect to the concatenation of node subsequences along a directed walk through the graph.
We do not require the alignment described by the edit string to start at the beginning of the sequence of the initial node, nor to terminate at the end of the sequence of the terminal node.
To allow the path model to support differences from the graph, each path is composed of a series of node mappings $p_i = m_1 \ldots m_{|p_i|}$ which are semantically identical to the alignment format used by standard aligners.
Each mapping $m_i = ( b_i, \Delta_i )$ has a starting position encoded as a node and offset in the graph $b_i = ( n_j, o_i )$ and a series of edits $\Delta_i = \delta_1 \ldots \delta_{|m_i|}$.
Edits $\delta_i = ( f_i, t_i, r_i )$ represent a length $f_i$ in the graph node $n_j$ (a ``from length'' in the reference), a length $t_i$ in the sequence the path encodes (a ``to length'' in the query), and encode an additional sequence $r_i$ that would replace the sequence at the given position in the reference in order to transform it into the query.
In the case of exact matches, we allow the replacement sequence $r_i$ to be empty.
They describe a function that transforms \emph{from} the graph \emph{to} the sequence.
Edits allow for rich encoding of alleles, but in practice alignments are rendered in terms of matches, mismatches, and indels.
We encode matches when $f_i = t_i \land r_i = \emptyset$, single mismatches when $f_i = t_i = 1 \land r_i \neq \emptyset$, deletions when $f_i > 0 \land t_i < 0 \land r_i = \emptyset$, and insertions when $f_i = 0 \land t_i > 0 \land r_i \neq \emptyset$.
As paths are described by a series of mappings with independent positions, they can represent all kinds of structural variation that can be represented in the bidirectional graph itself, including translocation, inversion, and copy number variation.
When mapping positions are always at the start of a node, the edit set for the path contains only matches, and the edges traversed by the path are all present in the graph, we say that the path is \emph{embedded}.
The paths from which we construct the variation graph are fully embedded, and in practice paths that contain differences occur only in the alignment of new sequences into the graph.

\subsection{Alignments}

Additional auxiliary information is important when analyzing collections of DNA sequencing data sets.
Each read has a name, and an identity related to a particular sequencing experiment.
It may be related to a particular genomic sample or individual.
DNA sequence reads themselves result from a previous set of analyses run on raw observations derived from DNA, perhaps fluorescence or current traces or images.
The process of collapsing this raw information into the sequence read yields a confidence in addition to a base call.
These are recorded in a quality string in FASTQ.
The need to collect this information has resulted in the development of the SAM/BAM sequence alignment format, which provides a standard for linking the called bases (sequence), quality information, read name, features of the alignment against a reference genome and additional optional freeform annotations.

I follow this same model in developing an alignment format for read alignments to the graph.
An aligned set of sequences $Q$, $A = a_1 \ldots a_{|Q|}$, represents a sequencing experiment.
Each aligned read connects a sequence, an (optional) quality string, a path through the graph, and an optional set of $D_i$ annotations: $a_i = (s_i, q_i, p_i, k_1\ldots k_{D_i})$.
In principle the read sequence can be reconstructed from the path, but retaining the sequence information makes the alignment object lossless with respect to the input FASTQ and provides redundancy which can help in data processing.
Alignments may thus be reprocessed without reference to the graph to which they were aligned.

At their heart alignments redundantly link a sequence with a path through the graph.
They form the basis for operations on the graph, as new sequences mapped into the graph may be used to extend the graph itself by editing the graph to include the sequences represented by the alignments.
Just as alignments describe a function to edit the graph, they can describe the relationship between the sequence space of two graphs.

\subsection{Translations}
\label{sec:translation}

A generalization of the alignment is the translation set $\Phi = \phi_1 \ldots \phi_{|\Phi|}$, which relates paths in different graphs to describe the alignment between them.
A translation $\phi = (p_f, p_t)$ defines the projection between two paths which may arise in the context of two graphs $G_f$ and $G_t$.
In this use each $p_f$ corresponds to a path relative to $G_f$ (conventionally the base or reference graph), and each $p_t$ to some path in $G_t$.
However, both paths in each pair could be relative to the same graph, in which case translations allow us to descibe collections of allelic differences to the base graph.

If each node in both graphs is represened in some graph translation in $\Phi$ then it provides an isomorphic relationship between the coordinates and sequences in both graphs.
Provided each $\Phi$ encodes an isomorphism, then we can layer a series of $\Phi_i$ together to provide a coherent coordinate space across any number of updates to a given graph.
Consider a function pattern $translate$, which allows the projection of paths relative to $G_f$ through translations $\Phi$ to yield paths in $G_t$: $translate(p_i, \Phi) \Rightarrow p_j \in G_t$, and similarly allows the transformation of a base graph into a target graph: $translate(G_f, \Phi) \Rightarrow G_t$.
If we have a series of $(G_i, \Phi_1) \ldots (G_\rho, \Phi_\rho)$, where $translate(G_i, \Phi_i) \Rightarrow G_{i+1}$ and thus each $\Phi_i$ describes an isomorphism between $G_i$ and $G_{i+1}$, then we can generate a graph translation $\Phi_\Delta$ providing $translate(G_1, \Phi_\Delta) \Rightarrow G_\rho$.
We build this graph translation with the function $layer(\Phi_\alpha, \Phi_\beta) \Rightarrow \Phi_{\alpha \cup \beta}$ by rewriting each path translation $\phi_i \in \Phi_\alpha$ so that its $p_t$ refers to $G_\beta$.
We do so by projecting the $p_t$ through $\Phi_\beta$, and finally adding any $\phi_j \in \Phi_\beta$ for which $p_f = \emptyset$, as these represent insertions of new sequence in $G_\beta$ relative to $G_\alpha$.

\subsection{Genotypes}
\label{sec:genotypes}

As path to path relationships can provide descriptions of allelic diversity, they form the basis for a graph-relative genotype encoding.
To represent the exact genotype of a particular sample with ploidy $\nu$ at a given locus $\iota$ we can simply collect the multiset of alleles $\pi_\iota = ( p_1 \ldots p_\nu)$.
We could alternatively build a probabilistic model $\varpi$ of an unphased genotype by using a set of $\mu$ alleles $\{ p_1 \ldots p_\mu\}$.
To do so we associate likelihoods $\gamma_\xi$ for each possible genotype $\pi_{\iota_\xi}$ that could be sampled from the alleles such that $\varpi = \gamma_1 \ldots \gamma_{\mu! \over{\nu!(\mu-\nu)!} }$.
In practice we develop our $\gamma_\xi$ out of quality information from the reads and a sampling model related to $\nu$ \cite{garrison2012haplotype,li2011stats}.
Given this definition of a genomic locus it is clear that existing genotyping models can be applied to drive genotyping using read sets aligned to the graph, and the output of the genotyper is defined fully in the space of the graph.

\subsection{Extending the graph}
\label{sec:extending}

Given an alignment $a_i$, we can edit the graph $G$ so that it includes the alignment and the sequence it represents as an embedded path, $augment(G, a_i) \Rightarrow (G', \Phi)$, such that $translate(p_i, \Phi) \in G'$.
To update the path space of the graph we project all paths, including that of $a_i$, through the translation implied by the augmentation of the graph with $p_i$.
Any other $a_j$ whose path $p_j$ overlaps $p_i$ would no longer be valid, although it could be projected through the graph translation $\Phi$ as well to express it in the space of the new graph $G'$.
Updating the graph one alignment at a time is inefficient as we need to build and layer a new translation for each alignment.
It is simpler to edit the graph in a single step, taking a collection of alignments and including them in the graph, $edit(G, A) \Rightarrow (G', \Phi)$.

One way to accomplish this is to first take the set of unique mappings represented in the paths of $A$, $\Omega = \{ m_1 \ldots m_{|\Omega|}\}$, and for each $m_i$ edit $n_i$ at the breakpoints where any new variation would need to be added in.
Then, walking through each alignment we add in unique novel sequences and their linkages to the preexisting nodes or new breakpoints to the graph.
This process will disrupt the identifier space of the nodes and edges of the graph, but it naturally yields a translation that can be used as described in section \ref{sec:translation}.
Note that both alignments and genotypes are based on paths, so this mechanism can be used to extend the graph based on any sequence level differences that we observe either through alignment or variant calling.

\section{Variation graph construction}

We will use variation graphs as the core model for a number of essential processes in genome inference.
This model can represent many graphical sequence models used in genomics.
Each one necessitates conversion to project it into the variation graph model.
Here I describe the transformation of a number of graphical models into variation graphs, including MSAs, assembly graphs, and alignment graphs induced from pairwise alignments.
In some cases the conversion is direct, but in others it requires the addition of new labels to our model.
Variation graphs may also be built from first principles, provided a function that aligns a sequence into the graph and the editing operations described in \ref{sec:extending}.

\subsection{Progressive alignment}
%*1p 1h*
Assume a function \emph{align} that takes a sequence $Q$ and a variation graph $G$ and yields the path $p$ that is the best model of $Q$ in $G$, such that it maximizes a $\lambda$-parameterized measure $score_\lambda(p)$: or $p = \arg\max_{score_\lambda(p)} [\forall p \in G : seq(p) = Q]$.
Provided $score_\lambda$ scores exact matches above mismatches and gaps, the alignment given by $p$ will be a kind of minimal description of $Q$ in the basis of $G$.
Equivalently, it is a $\lambda$-minimal augmentation of $G$ required to include $Q$.

We can build a variation graph progressively from a set of sequences.
If we have a series of $k$ queries $q_1 \ldots q_k$, then we can build a progressive alignment by a series of edit and alignment operations applied to the variation graph.
First, take the empty graph $G_\emptyset$, to which any alignment will yield a path $p_1$ that has no mappings an which encodes the query sequence $q_1$ as a replacement sequence in the path.
We then edit the graph to add the sequence using $edit(G_\emptyset, p_1) \Rightarrow G_1$.
For each subsequent $q_j$ we obtain the next graph by finding the alignment $align(q_j, G_j) \Rightarrow p_j$ and editing the graph with it to yield the next graph $edit(G_j, p_j) \Rightarrow G_{j+1}$ until $j = k$ and we obtain our final graph $G_{\cup \forall q_i}$.
This simple approach is attractive as it allows the variation graphs to be built from whole sequences using only techniques that are native to the variation graph model itself.
I later describe an implementation of this process multiple sequence graph alignment, {\tt vg msga}.

\subsection{Using variants in VCF format}
%*1.5p 1h*
As discussed in section \ref{sec:seq_dag_vcf}, the VCF format that is popular in resequencing implies a sequence DAG.
We can consider how to build a variation graph using the core operation $edit$.
First, we build a variation graph from the reference genome $Q_\textbf{ref}$ : $G_\textbf{ref}$.
This graph contains one path $p_\textbf{ref}$ : $seq(p_\textbf{ref}) = Q_\textbf{ref}$.
As described in section \ref{sec:genotypes} each locus reported in VCF can be encoded as a set of paths $P_\textbf{vcf} = p_1 \ldots p_V$, each representing a different one of the $V$ alleles in the VCF.
We now edit the graph to embed these allele paths, $edit(G_\textbf{ref}, P_\textbf{vcf}) \Rightarrow G_\textbf{vcf}$.
Note that the graph has losslessly encoded the variant input and reference genome, and it is possible to regenerate the VCF file input by walking the positions of $p_\textbf{ref}$ and enumerating the overlapping paths as alleles in VCF format.

For efficiency, we have not implemented VCF to variation graph conversion with specifically this algorithm, but instead build up $G_\textbf{vcf}$ by walking along the reference genome $Q_\textbf{ref}$ and processing each locus independently.
This exploits the partially ordered property of the VCF to limit memory requirements when working with whole genome graphs.
For regions before, after, and between variant records at reference offsets $i$ and $j$ we add a node $n_\textbf{curr} : seq(n_\textbf{curr}) = substr(Q_\textbf{ref}, i, j)$, linking these by edges to those nodes ending at position $i$ of the reference and adding corresponding mappings for the reference path to $p_\textbf{ref}$.
At simple variant sites we add each of the alleles as a new node $n_{\textbf{var}_i}$, including an edge for each $e_{\textbf{curr} \prec \textbf{var}_i}$.
Here we also handle the reference allele differently in that we append a mapping $m_\textbf{ref} = ((n_\textbf{ref}, 0), \emptyset)$ to $p_\textbf{ref}$.
This approach seamlessly handles all kinds of common variation, including SNPs and indels, as well as more complex variation encoded as small haplotypes.

As long as the VCF records are ordered, this process allows for streaming conversion of the VCF format into a variation graph.
However, VCFs used to represent structural variation often do not describe a fully-ordered series of loci.
For instance, a large deletion may be described in one record, and followed by a number of records describing variation on the reference within the scope of the deletion.
In the graph, this results in the nesting of bubbles, and requires a deviation from a simple streaming algorithm in order to be handled.
Deletions must be recorded and linked into the downstream portion of the graph as it is later generated.
The reference allele for the deletion can only be defined once we have constructed the variation graph under the deletion.

VCFs may also encode phased haplotypes, which, like the reference, have a natural representation in the graph as paths.
Parsing these may require multiple passes over the VCF due to the memory requirements for storing large numbers of haplotypes uncompressed and cross-indexed to allow traversal in RAM.
To prevent $O(H|G_\textbf{vcf}|)$ growth of the required memory to store these, we implement efficient compression strategies on the haplotype set that exploit their repetitiveness.
There is no semantic requirement that the encoded haplotypes in VCF are valid, which introduces some complexity in the implementation of this method.
We must break haplotypes where they are found to be invalid in order to record them in VG format.
For instance, a phased VCF may report more than $\nu$ (expected ploidy) alleles for a given individual, such as when deletion and SNP variants overlap.
We expect these haplotype paths to be fully embedded in the graph an they can be represented as a walk through nodes.
Although haplotype sets are equivalent to large collections of paths, we term them \emph{threads} to indicate that they have a simpler representation than full paths.

\subsection{From gene models} % (GFF)

A reference-based RNA splicing graph is usually expressed as a set of alignments, perhaps as named intervals in BED or the General Feature Format (GFF).
As with the generic VCF generation algorithm, we can convert the transcripts to alignments $A$ relative to the graph.
Then we can then embed the transcript paths in the graph $edit(G_\textbf{ref}, A) \Rightarrow G_\textbf{splice}$.
Any transcript in our set is thus encoded by the graph, and can be matched to it directly with alignment.
We also allow for novel isoforms with local similarity to the known set.

\subsection{From multiple sequence alignments}
%*0.5p 0.5h*
A multiple sequence alignment in matrix form has a simple translation into a sequence DAG and thus a VG \cite{lee2002POA}.
Given a set of sequences $Q = q_1 \ldots q_\kappa$ their $\upsilon$-long mutual alignment may be described in a $\kappa \upsilon$ matrix $X$ designed to maximize $\sum_{i=1}^{\upsilon} \sum_{j=1}^{\kappa} \sum_{k=1}^{\kappa} \delta_{X_{ij}X_{ik}}$, where $\delta$ is the Kronecker delta.
The alphabet used to encode the matrix is the same as the input sequences with the addition of a special gap character $\Box \neq \Box$, and gaps thus do not contribute positively to the matrix score.
To build a variation graph $G_\textbf{msa}$ from the MSA we proceed from $i = 1 \to \upsilon$ through $X$.
For each unique character ${\cal B}$ in the query alphabet $\Sigma \setminus \Box$ found in each row $i$, we create a node $n_{\cal B}$ in $G_\textbf{msa}$ and append a mapping to each path $p_i$ for which $n_{\cal B} \in q_i$.
We construct the edges of $G_\textbf{msa}$ by taking the distinct pairs of consecutive node traversals found in the path set $P_\textbf{msa}$ produced after the generation of the nodes in MSA traversal.
For each pair connecting nodes $n_i$ and $n_j$ we add an edge $e_{ij}$.
We can optionally compact series of nodes (which represent single characters) where no bifurcations occur, which is similar to the operation used to compact DBGs to yield unitig graphs.

Instead of a matrix, we can formulate the multiple alignment as an alignment graph (described in section \ref{sec:genome_alignment_graphs}).
By making this graph bidirectional, and thus equivalent in information content with the Enredo graph, we can see that it is equivalent to a variation graph.
Aligners that produce data formats of this type, such as Cactus \cite{Paten:2011fva}, can thus be used to produce VGs, so long as the relationship between the input sequences and the graph is recorded and can be converted into a path description.
I later use this technique to build a pangenomic reference system for a diverse set of yeast strains.

\subsection{From overlap assembly and deBruijn graphs}
%*2p 1.5h*

Overlap-based sequence graphs used in assembly, described in sections \ref{sec:overlap_graphs}, \ref{sec:de_bruijn_graphs}, and \ref{sec:string_graphs}, are nearly identical to variation graphs.
The critical difference between these models and the variation graph is that they attach a label to each edge describing the alignment between the pair of nodes which they connect.
Variation graphs do not support such a feature in their basic definition, as it is unimportant for any use of variation graphs besides temporarily representing overlap graphs.
This process is known as \emph{bluntification}, as the edges lose information and sequences given by traversing the graph are represented directly in the concatenation of node sequences.

If we reduce an overlap between a pair of nodes, it will render other overlaps on the same nodes incorrect.
Thus it is essential that the bluntification algorithm work by the reduction of sets of overlaps on edges which are transitively closed by connection to the same ends of each node, $net(e_{ij}) \Rightarrow \forall e_{i*} \in G \cup \forall e_{*j} \in G$, considering both strands of the graph when doing so.
For each net we apply a function $pinch(net(e_{ij})) \Rightarrow G_{\textbf{pinch}_{ij}}$, which reduces the overlaps between the nodes in the net into a blunt-edged variation graph.
We then link $G_{\textbf{pinch}_{ij}}$ back into the rest of the graph by connecting with the inbound links to each node involved in the net.
We may thus rewrite the input overlap graph as a variation graph.

In a de Bruijn graph or string graph such as SGA or Fermi2, overlaps are defined given only a length.
This simplifies the implementation of $pinch$, as no further computation is required to correctly determine the mutual alignment of overlapping sequences.
In contrast, overlaps in a generic string or overlap graph are correctly defined as alignments.
Resolving a single pairwise alignment into structures in the graph is trivial, but it becomes considerably more complex when many sequences map into a transitively closed set of overlaps.
These nets can then be resolved into an alignment graph by an algorithm similar to that given in section \ref{sec:from_pairwise_alignments}, but in practice, {\tt vg} implements bluntification by conversion of the input graph to a pinch graph library that was used to resolve the same problem in whole genome alignment \cite{Paten:2011fva}.

\subsection{From pairwise alignments}
\label{sec:from_pairwise_alignments}

Although they are mathematically equivalent, the complexity of converting various graphical genome models to variation graphs can be considerable, and {\tt vg} still lacks an ideal solution to the problem.
A set of pairwise alignments imply a variation graph, however I know of no contained method that will generate the variation graph or lossless string graph from these alignments.
One apparent problem is the memory requirements of doing so.
The full string graph for a noisy or divergent sequence set is rarely possible to maintain in RAM.
This suggest that an external memory based approach may enable greater applicability of such an approach.
To explore this, I developed an algorithm to do so that operates in external memory, which I here present in detail.
It operates by conversion of the alignment set into an alignment graph and the subsequent use of this graph in the elaboration of the variation graph including paths representing the input sequences.
The resulting graph is a lossless representation of the input and the alignments between them.
To distinguish the approach from string graphs, which imply error correction, I call this variation graph induction model the \emph{squish graph}.

% seqwish algoritm
{\tt seqwish}\footnote{\url{https://github.com/ekg/seqwish}} implements a lossless conversion from pairwise alignments between sequences to a variation graph encoding the sequences and their alignments.
As input we typically take all-versus-all alignments, but the exact structure of the alignment set may be defined in an application specific way.
This algorithm uses a series of disk-backed sorts and passes over the alignment and sequence inputs to allow the graph to be constructed from very large inputs that are commonly encountered when working with large numbers of noisy input sequences. 
Memory usage during construction and traversal is limited by the use of sorted disk-backed arrays and succinct rank/select dictionaries to record a queryable version of the graph.

%## squish graph induction algorithm

As input we have $Q$, which is a concatenation of the sequences from which we will build the graph.
We build an compressed suffix array (CSA) mapping sequence names to offsets in $Q$, and also the inverse using a rank/select dictionary on a bitvector marking the starts of sequences in $Q$.
This allows us to map between positions in the sequences of $Q$, which is the format in which alignment algorithms typically express alignments, and positions in $Q$ itself, which is the coordinate space we will use as a basis for the generation of our graph.
To relate the sequences in $Q$ to each other we apply a function $map$ to generate alignments $A$.
Although these alignments tend to be represented using oriented interval pairs in $Q$, for simplicity and robustness to graph complexity, we describe $A$ as a vector of pairs of bidirectional positions (sequence offsets and strands) $b$ in $Q$ , such that $A = (b_{q}, b_{r}), \ldots $.
We sort $A$ by the first member ($b_{q}$) of each pair, ensuring that the entries in $A$ are ordered according to their order in $Q$.

To query the induced graph we build a rank/select dictionary allowing efficient traversal of $A$, based on a bit vector $A_{bv}$ of the same length as $A$ such that we record a 1 at those positions which correspond to the first instance of a given $b_{q}$ and record a 0 in $A_{bv}$ otherwise. 
We record which $b_{q}$ we have processed in the bitvector $Q_{seen}$ which is of the same length as $Q$.
This allows us to avoid a quadratic penalty in the order of the size of the transitive closures in $Q$ given by the $map$ function.

Now we inductively derive the graph implied by the alignments.
For each base $b_{q}$ in $Q$, we find its transitive closure $c_{q} := [b_{q}, b_{r_{1}}, \ldots ]$ via the $map$ operation by traversing the aligned base pairs recorded in $A$.
We write the character of the base $b_{q}$ to a vector $S$, then for each $b_{c}$ in $c_{q}$ we record a pair $[s_{i}, b_{c}]$ into $N$ and its reverse, $[b_{c}, s_{i}]$ into $P$.
We mark $Q_{seen}$ for each base in each emitted cluster, and we do not consider marked bases in subsequent transitive closures.
By sorting $N$ and $P$ by their first entries, we can build rank/select dictionaries on them akin to that we built on $A$ that allow random access by graph base (as given in $S$) or input base (as given in $Q$).

To fully induce the variation graph we need to establish the links between bases in $S$ that would be required for us to find any sequence in the input as a walk through the graph.
We do so by rewriting $Q$ (in both the forward and reverse orientation) in terms of pairs of bases in $S$, then sorting the resulting pairs by their first element, which yields $L = [(b_{a}, b_{b}), \ldots ]$.
These pairs record the links and their frequencies, which we can emit or filter (such as by frequency) as needed given particular applications.
In typical use we take the graph to be given by the unique elements of $L$.

Our data model encodes the graph using single-base nodes, but often downstream use requires identifying nodes and thus we benefit from compressing the unitigs of the graph into single nodes, which reduces memory used by identifiers in analysis.
We can compress the node space of the graph by traversing $S$, and for each base querying the inbound links.
Maintaining a bitvector $S_{id}$ of length equal to $S$ we mark each base at which we see any link other than one from or to the previous base on the forward or reverse strand, or at bases where we have no incoming links.
By building a rank/select dictionary on $S_{id}$ we can assign a smaller set of node ids to the sequence space of the graph.

Given the id space encoded by $S_{id}$ we can materialize the graph in a variety of interchange formats, or provide id-based interfaces to the indexed squish graph.
To generate graphs in {\tt vg}  or GFA format, we want to decompose the graph into its nodes ($S$), edges ($L$) and paths ($P$).
The nodes are given by $S$ and $S_{id}$, and similarly we project $L$ and $P$ through $S_{id}$ to obtain a compressed variation graph.


\section{Data interchange}
%*0.5p 0.5h*

{\tt vg}'s design largely preceeded its development.
The mathematical concepts described in the previous section lie at the core of its implementation.
A schema language, Google Protocol Buffers (Protobuf), is used to define a compact description of data structures sufficient for the representation of all the required data models.
One cause of this pattern was my involvement in the GA4GH-DWG at the beginning of my thesis.
As the GA4GH is chartered to encourage focus on data interchange, the GA4GH-DWG was then seeking a coherent way of describing graph genomes\footnote{\url{https://github.com/ga4gh/ga4gh-schemas}} to allow support pangenomic principles resequencing.
I thus implemented the schema for {\tt vg} in the popular Protobuf schema language.
This provided a core API on which to bulid {\tt vg}.
It also implied a set of streaming data formats, which I implemented as a template library capable of serializing any stream of Protobuf objects.
Due to the reliance on Protobuf, the only code needed to implement reading and writing of these formats is {\tt vg} schema and the stream library\footnote{At the time of writing the schema, \url{https://github.com/vgteam/vg/blob/master/src/vg.proto} and stream parsing library \url{https://github.com/vgteam/vg/blob/master/src/stream.hpp} total around 1000 LOC, and are sufficient to link any C++ program into the {\tt vg} ecosystem.}.
This greatly simplified the process of developing libraries for working with the variation graph data models.
Although in practice the Protobuf data structures are slower to parse than handmade C-struct serializations like BAM, the amount of effort required to begin writing efficient and structured binary data formats was considerably less with the schema based approach.
Most importantly, the schema based definition of the core data types in {\tt vg} helped new developers and researchers using the system quickly appreciate the basic concepts.
It is not easy to measure, but I believe this has been a key cause of the success of the project and unifying many different research interests spread across many groups.

Several data formats have been important to {\tt vg}.
The graph itself can be serialized in non-overlapping chunks, where each edge $e_{ij}$ is stored once in the chunk $G_\textbf{chunk} : n_i \in G_\textbf{chunk}$.
Path mappings must have a rank that identifies their position in the path in order to be subdivided in this way, and otherwise must be serialized literally in a single vector as in GFA's ``P'' namespace.
This allows them to be read in and rebuilt even if they have been serialized out of order.
A series of alignment objects is a sensible output of the mapping algorithm {\tt vg map}.
The file format produced by writing out a series of Protobuf alignment object serializations using the {\tt stream.hpp} library is called GAM, for Graphical Alignment/Map, in analogy to SAM (Sequence Alignment/Map format).

These data models have various other equivalent serializations.
As mentioned before, the GFA format can be used to directly encode VGs, although some of the default fields (particularly the overlap CIGAR on edges) are generally not useful and carry no information.
However, GFA lacks a representation of an alignment with the same semantics as that described here.
VCF files and the reference they refer to can encode VGs, and VGs of a partially ordered format can be deconstructed into the equivalent VCF.
As paths in variation graphs can be used to represent any kind of existing annotations, data providers who represent annotations across many genomes (such as ENSEMBL Genomes) can build their annotation sets which were previously spread across many genomes into a single one embedded in a VG.
To enable this several collaborators\footnote{Jerven Bolleman and Toshiaki Katayama among others.} have developed a Resource Description Framework (RDF) compatible version of the core VG model.

\section{Index structures}

Alignment is the fundamental operation in resequencing.
Given a sequence (the ``query'') and a reference system (the ``target''), we infer a description of the query in terms of edits to the reference.
Variant calling and other genome inference problems rely on these alignments.
As described in section \ref{sec:sequence_alignment}, the large collections of read data produced by current sequencing methods require efficient read alignment to support downstream analysis.
Typically, these methods develop indexes of their reference genome, using $k$-mer hash tables or FM-index/CSA based methods capable of arbitrary-length match inference.
These indexes remain static during the resequencing analysis, and can thus be made very compact and designed to support efficient queries.
%{\tt vg} uses 

When the genome is just a linear string, distances between locations may be computed trivially, and subsets of the sequence are simply substring operations on the vector representing the genome, so no additional structure is required to seed the alignment of reads to the genome.
However, this situation changes in graphs, where the computation of distances becomes greatly more complex and particular topologies of the graph must be recorded and reproduced.
Here, I will show that graph distances may be estimated using an approximate sort and the paths embedded in the graph.
However, to do so requires efficient indexes of the path structure in the graph.
Furthermore, loading the entire graph into memory in a na\"{i}ve manner can be very expensive, and effort is required to minimize the runtime costs so as to enable resequencing even on lower-memory commodity systems.
%As with 
% summarize succinct structures for path queries

\subsection{Dynamic in-memory graph model}

Serialized in .vg or compressed GFA format, the graph of the 1000GP is not much larger than the uncompressed human reference genome.
However, the performace-oriented implemenation of the dynamic variation graph which I developed at the beginning of my studies can use a hundred times this mich memory when the entire graph is loaded into RAM.
In this scheme implemented in {\tt vg}, indexes on the node idientifier space of the graph allow for fast traversal and query of nodes by identity and neighborhood.
Various inefficiencies are accepted, such as on the hash table occupancies used to build these indexes, in the pursuit of higher performance during dynamic modification of the graph.
I now believe that it should be easy to provide a dynamic VG in low memory by using a succinct encoding, but I have not yet completed any work on this issue.
Operating on graphs of hundreds of millions of nodes with annotations like paths remains a difficult problem in this and other fields.
In most cases the graph can be subdivided (as with map/reduce processing patterns \cite{dean2008mapreduce} which underpin most industrial operation on large graphs \cite{cohen2009graph}).
This is particularly easy to implement for sequence DAG VGs, allowing for us to work on graphs of arbitrary sizes if they are approximately linear.

\subsection{Graph topology index}
\label{sec:graph_topology_index}
%*2p 3h*
While a dynamic in-memory model of a dense VG may be difficult to scale, it is comparatively easy to build succinct indexes of the graph that use only a small constant factor more memory than the uncompressed graph, yet provide support for a number of queries which are important for read mapping.

When using a variation graph as a reference system, we are unlikely to need to modify it.
As such we can compress it into a system that provides efficient access to important aspects of the graph.
Specifically, we care about the node and edge structure of the graph and queries that allow us to extract and seek to positions in embedded paths.
We would like to be able to query a part of the graph corresponding to a particular region of a chromosome in a reference path embedded in the graph.
Similarly, if we find an exact match on the graph using GCSA2, we would like to load that region of the graph into memory for efficient local alignment.

We implement a succinct representation of variation graphs in the XG\footnote{``X'' implies compression and ``G'' refers to the graph that is compressed.} library, using data structures from \href{https://github.com/simongog/sdsl-lite}{SDSL-lite}.
Node labels and node ids are stored in a collection of succinct vectors, augmented by rank/select dictionaries that allow the lookup of node sequences and node ids.
An internal node rank is given for each node, and we map from and to this internal coordinate system using a compressed integer vector of the same order as the node id range of the graph we have indexed.
To allow efficient exploration of the graph, we store each node's edge context in a structured manner in an integer vector, into which we can jump via a rank/select dictionary keyed by node rank in the graph.
Efficient traversal of the graph's topology via this structure is enabled by storing edges a relative offsets to ``to'' or ``from'' node, which obviates the need for secondary lookups and reduces the cost of step-wise traversal to member access on the containing vector and the cost of parsing each node context record that we encounter.
Paths provided to XG are used to induce multifarious coordinate systems over the graph. 
We store them using a collection of integer vectors and rank/select dictionaries that allow for efficient queries of the paths at or near a given graph position, as well as queries that give us the graph context near a given path position.

An XG index of $G = (N, E, P)$ is comprised primarily of the backing graph vector $G_\textbf{iv} = g_1 \ldots g_{|N|}$, with each $g_i$ recording the edge context for node $n_i$ in the graph: $g_i = ( \eta_i, \Xi_i)$, a sequence vector $S_\textbf{iv}$ recording the sequences labels of the nodes in a bitcompressed form, and a path membership mapping $N_\textbf{path}$.
Each $p_i \in P$ is encoded with a set of structures that allow random access of the graph by path position, which is important for the use of paths as reference coordinate systems in the graph.

To enable better compression, the node sequence space is recorded as a concatenation of node labels $S_\textbf{iv} = seq(n_i) \ldots seq(n_{|N|})$, in which each node has an offset in this sequence space defined by $seq_\textbf{offset}(n_i)$.
In bitvector $S_\textbf{bv} : |S_\textbf{bv}| = |S_\textbf{iv}|$ we set 1 at each first character in a node label, and 0 otherwise: $S_\textbf{bv}[i] = 1 \iff \exists j : seq_\textbf{offset}(n_j) = i \lor 0$.
Random access by node rank $i$ is provided by function $S_\textbf{bv}^{select_1}$, allowing us to find the sequence given a node rank in $G_\textbf{iv}$.

To allow node identifiers to be separated from the internal rank in $G_\textbf{iv}$, each context header $\eta_i$ records the external id of $n_i$.
This supports operation on subsets of larger graphs, as the subgraph retains a mapping into the node identifier space of the larger graph.

The $G_\textbf{iv}$ node context header $\eta_i$ thus links the topology of the graph to node sequences and externally resolvable node ids, as well as recording the number of inbound and outbound edges in the graph context of
$n_i$: $\eta_i = [ id(i), seq_\textbf{offset}(n_i), |seq(n_i)|, |\{e\}| : \exists e_{i*} \lor \exists e_{\overline{i}*}, |\{e\}| : \exists e_{*i} \lor e_{*\overline{i}} ]$.
Meanwhile, the edge context $\Xi_i$ enumerates the set of edges that connect to this node in a structured way that allows for oriented traversals across the two strands of the graph: $\Xi_i = \bigcup_{j=1}^{|N|} \forall e \in e_{ij}, e_{\overline{i}j}, e_{i\overline{j}}, e_{\overline{ij}}, e_{ji}, e_{\overline{j}i}, e_{j\overline{i}}, e_{\overline{ji}} : \exists e$.
To enable fast traversal we rewrite the edges in terms of relative positions in the encoded $G_\textbf{iv}$ vector, which is stored as a bitcompressed integer vector using SDSL-lite's template primitives.
Each node record is stored contiguously.
We delimit the records by a secondary bitvector $G_\textbf{bv}$, for which we build supports for functions $G_\textbf{bv}^{rank_1}$ and $G_\textbf{bv}^{select_1}$, which allow random access of $G_\textbf{iv}$ by node $id$.
Previous designs decomposed the graph structure into a set of parallel vectors, but this required multiple select queries during traversal and provided poor cache locality and performance.

Node to path membership recorded in $N_\textbf{path} = ( p_1 \ldots p_{|\{p_i : n_1 \in p_i \}|} ) \ldots ( p_{|N|} \ldots p_{|\{p_i : n_{|N|} \in p_i \}|} )$, and random access to this integer vector is provided by a rank/select dictionary built on a bitvector delimiting the various node path membership lists.
Nodes with no path membership are marked in $N_\textbf{path}$ with a 0.

Each path is represented in a set of succinct data stuctures that let us walk the path by starting at a particular node, query the path position of a given node, or find the node at a particular path position.
We store the path $p_i = m_1 \ldots m_{|p_i|}$ by decomposing its mappings into the nodes (id and orientation) it traverses $p_{i_\textbf{ids}} = id(n_j) \forall n_j \in p_i$ and $p_{i_\textbf{dir}}$ such that $p_{i_\textbf{dir}}[j] = 0 \iff m_j = (n_j, \ldots) \lor 1 \iff m_j = (\overline{n_j}, \ldots)$.
To allow rank and select queries on the node ids, we can store $p_{i_\textbf{ids}}$ in a wavelet tree, although in practice performance is greatly improved by also recording a minimum node id, so as to decrease the alphabet size of the wavelet tree, which has a large effect on WT size.
So that we may transform nodes to path positions, we use an integer vector to store a path position for each node traversal in the path, $p_{i_\textbf{pos}}$.
To go from path offset to node, we build a bitvector that marks the beginning of each node traversal in the path in a manner similar to that used to mark the sequence beginning of each node in $S_\textbf{iv}$, such that $p_{i_\textbf{offset}}$ is a bitvector of length $\sum_{j=1}^{|p_i|} |seq(p_i[j])|$ where we have marked 1 for each node start in the path, $p_{i_\textbf{offset}}[j] = \left( 1 \iff \exists n_k \in p : j = \sum_{m=1}^{k} |seq(p_i[m])| \right) \lor 0$.
By implementing $p_{i_\textbf{offset}}^{rank_1}$ we can find the node at a given path position $\mathcal{Q}$ by $p_{i_\textbf{offset}}^{rank_1}(\mathcal{Q}) \Rightarrow j : \left( \sum_{k=1}^{j} |seq(p_i[k])| \leq \mathcal{Q} \land \sum_{k=1}^{j+1} |seq(p_i[k])| > \mathcal{Q} \right)$.
We can also find the position of the $j$th node in a path as $p_{i_\textbf{offset}}^{select_1}(j)$.
A CSA and rank dictionary $P_\textbf{csa}$ and $P_\textbf{name}^{rank_1}$ map from path name to internal rank of the path in $P$, which ensures maximum scalability of the path set names.

A number of compression techniques can be applied to the data models in XG to reduce the size of the overall index without any loss in functionality.
However, many of these compression methods producing compressed bitvectors, integer vectors, and wavelet trees, will result in slower access.
In the context of a short read aligner, such losses may be undesirable as long as there is sufficient memory to load the entire index into system memory, and so I have tuned the index by chosing compression strategies appropriate for its use on current datasets.

\subsection{Graph sequence indexes}
\label{sec:graph_sequence_indexes}
%*1p 2h*
Indexing the sequence space of a variation graph can be achieved using the same $k$-mer based techniques that are popular in pairwise alignment.
In the early stages of this project, I implemented a $k$-mer based index using a disk-backed system (see section \ref{sec:generic_disk_backed_indexes}).
An efficiently-built in-memory index of $k$-mers, perhaps sampled and possibly $w,k$-minimizers \cite{marccais2018asymptotically,li2018minimap2}, appears to be a reasonable basis for a sequence to graph aligner, but I did not explore this beyond these initial experiments.
Our research focus on the replication of results of the MEM-seeded short read mapper {\tt bwa mem} encouraged a similar approach for {\tt vg}.
A suffix tree provides access to a wide range of efficient sequence matching and inference algorithms, and provides a strong basis for a sensitive read aligner.

On seeing my early experiments, Jouni Sir\'{e}n, who had recently joined our lab, suggested that there might be a way to adjust his GCSA index model to work on arbitrary graphs.
He proposed a DBG transformation of the graph as input to the indexing process.
The resulting indexing model is similar to both GCSA, in its encoding of the graph topology, and succinct DBGs, which are also developed from BWTs based on a $k$-mer set.
Sir\'{e}n's design allows for the use of much longer $k$-mers than are typcially considered in DBGs, with the final DBG to be indexed having $k=256$.
As few contemparary reads are likely to generate 256bp-long sequences with no mismatch from the reference\footnote{Illumina's reads rarely reach 256bp, and when they do they tend to have higher error rates in the later cycles. The 15\% error rate of PacBio and ONT sequencing mean that a 256bp exact match is extremely unlikely, although PacBio circular consensus reads (CCR) may approach this level of accuracy.}, this approach effectively allows us to find all the SMEMs for a typical sequencing read.
The addition of the LCP array to encode the suffix tree alongside GCSA2's BWT effectively generalized all essential operations possible on a suffix tree of a string to the GCSA2, and enabled us to find SMEMs and develop sensitive MEM-based seeding heuristics using the index.
The design of GCSA2 is wholly Sir\'{e}n's work \cite{siren2017indexing}, and it forms one of the core components of the {\tt vg} mapper paper, which we coauthored.
I present it here as the mapping algorithm I developed for {\tt vg} relies on it heavily.

In broadest terms, GCSA2 employs a de Bruijn graph as a $k$-mer index of a variation graph.
Effectively the DBG nodes are labeled with their positions in the variation graph from which the DBG is generated.
To reduce redundancy, GCSA2 uses a \emph{pruned} DBG by using nodes with labels shorter than $k$ where these shorter strings still uniquely identify the beginning of their coresponding paths in the unpruned DBG.

This pruned graph is then encoded in a generalization of the FM-index built on a BWT of the $k$-mers in the pruned DBG.
The FM-index is extended to support search over the DBG by adding auxiliary bitvectors which record the indegree and outdegree of nodes in the pruned DBG.
These are used in backwards search to adjust the suffix array ranges that are considered, expanding them when the indegree is greater than 1 and contracting them when the out degree is greater than 1, in effect encoding the graph structure into the index in a space efficient and traversal optimized manner.
The technique is the same as that used in GCSA, although GCSA indexes a reverse deterministic automaton rather than a DBG.
Impressively, GCSA2 uses around 1 bit per $k$-mer in indexes of the 1000GP pangenome graph for $k=128$, which is favorable with comparsion to true DBG indexes, as these need to record information about each $k$-mer that is not important for GCSA2.

As an FM-index like structure, GCSA2 supports queries for a pattern $X$ that yield the suffix array interval matching a given pattern: $find(X) = ( sp_{X}, ep_{X} )$, and $locate( sp_{X}, ep_{X} ) = \{ b_1 \ldots b_{count(sp_{X}, ep_{X})} \}$ which yields the positions in the input VG where the given patterns occur.
The afformentioned LCP array allows the index to support several operations that require a suffix tree, including $count(sp_i, ep_i)$, which returns the number of matches for a given suffix array range, and $parent(sp_i, ep_i) = (sp_j, ep_j)$, which conceptually allows us to traverse the suffix links embedded in the suffix tree and forms the core of maximal exact match (MEM) inference using GCSA2.

GCSA2 resolves a number of problems that limit the utility of other graph path indexing schemes.
Indexing a pruned DBG allows GCSA2 to be applied to arbitrary bidirectional string graphs.
It avoids exponential costs during backwards search using the graph augmentation of the FM-index, which is itself supported by the transformation of the input variation graph into a pruned DBG.
However, it does incur exponential costs in indexing, but to mitigate the affects of these it may be constructed mostly in external memory, and furthermore construction may be broken into pieces by chromosome or some other subdivision of the graph.
Its external memory construction algorithm uses a prefix doubling approach to transform an input DBG of order $k$ to one of order $2^mk$ through $m$ doubling steps.
The high-order DBG need never be loaded into RAM but is induced from a lower-order DBG augmented with sufficient positional context on its nodes to support the disk-backed prefix doubling process.
By default, {\tt vg} runs GCSA2 with input $k=16$ and $m=4$, yielding a final DBG with $k=256$.
In indexing a DBG, GCSA2 limits the maximum-length query for which no false-positive results will be returned to the order $k$ of the DBG.
This has the benefit of allowing GCSA2 to index complex graphs, as a shorter $k$ may be chosen to limit exponential costs.
Provided the final alignment is completed against the full VG using DP-based alignment techinques that have non-exponential costs, the DBG input to GCSA2 may be constructed from a VG in which complex regions have been masked or reduced in local complexity using prior information such as a haplotype panel.

To appreciate the costs of indexing\footnote{Precise sizes are given later in the discussion of results.}, the order-256 GCSA2 may be constucted using less than 500GB of scratch space, to and from which are written approximately 3TB during construction, all while requiring less than 50G of RAM.
This puts GCSA2 indexing's resource requirements well within the specifications of standard commodity compute servers.
The resulting index occupies between five and ten times the input graph's serialized size, so no more than 50G for something in the order of a human genome.

\subsection{Haplotype indexes}
%*1.5p 3h*
% worth mentioning gpBWT and PBWT
Recording the path set of a graph in XG (described in section \ref{sec:graph_topology_index}) requires $O(N{\cal L}|P|)$ space where ${\cal L}$ is the average haplotype length, $N$ is the number of nodes in the graph, and $|P|$ is the number of paths in the graph.
Although, such a representation can be compressed, the size of this representation will grow linearly with the addition of new paths, making it impractical as a means to record very large numbers of genomes.
This encoding does simplify positional queries, but to do so each path must be stored independently and without reference to other paths.

Recording collections of paths is an important requirement for the use of VG in resequencing, as the markovian property of the bare sequence graph $G = (N, E, P= \emptyset)$ means it can encode exponentially many paths relative to the true input path set used to build the graph.
This introduces significant issues during read mapping and genome inference.
With increasing variant density the number of haplotypes grows exponentially, and this can lead to spurious mismapping as discussed in section \ref{sec:1000GP_sim}.
The exponential growth of the path space of the graph has relevance for sequence indexing with GCSA2, and as described in \label{sec:graph_sequence_indexes}, simplification of the graph input to GCSA2 indexing is required to reliably build indexes.
A haplotype index is the only way that the pruning operation applied to the graph may preserve known haplotypes, rather than defaulting to the reference genome in such cases.
Efficient path indexes could be used for many operations in variant calling and phasing, and may have utility in assembly problems, for instance to losslessly record a read set embedded in a squish graph representing their mutual alignment (section \ref{sec:from_pairwise_alignments}).

Haplotypes from the genomes of the same species are often shared exactly, which suggests that they may be very efficiently compressed.
This property was first used to store large haplotype sets in the \emph{positional BWT} (PBWT) \cite{durbin2014efficient}.
As input the PBWT assumes a set of haplotype strings $S_1 \ldots S_m$ of the same length which describe a set of haplotypes relative to a set of variable loci.
$S_j[i]$ records the allele in haplotype $j$ found at locus $i$.
We set $S_j[i] = 0$ when haplotype $j$ has the reference allele at locus $i$, and $S_j[i] > 0$ if it encodes one of possibly several alternate alleles.
The PBWT can be understood as an FM-index of texts $T_1 \ldots T_m : T_j[i] = (i, S_j[i])$ \cite{gagie2017wheeler}.
To search for a haplotype of $h$ in the range $[i,j]$ we look for pattern $h' = (i, h[1]) \ldots (j, h[|h|])$.
The alphabet size of this FM-index is large, but the matrix like structure of the haplotype set means that we can implicitly encode the array indexes by building a separate sub index for each position.
Applying run length encoding to the BWT allows extremely good compression of real haplotype sets.

With the \emph{graph positional Burrows–Wheeler transform} (gPBWT) \cite{Novak2016gPBWT}, we extended this model to work on variation graphs.
The basic model is the same as the generic PBWT except that instead of variant matrix positions we consider haplotype traversals of node ``sides'' $n_i$ or $\overline{n_i}$, and rather than a local alphabet of variant alleles we encode a local alphabet $\Sigma_{n_i} = \{ j \in N | e_{ij} \in E \}$ which describes the set of nodes $n_{\{j \in N | e_{ij} \in E \}}$ to which haplotypes go after the current node (side) $n_i$.
As in the generic PBWT we build an FM-index of $T_1 \ldots T_m$, encoded in what we call the $B_s$ arrays, which provide the local description of prefix sorted haplotypes (equivalently, threads) traversing each node side $n_s$.
To deal with bidirectionality of paths in variation graphs, each haplotype must be encoded in its forward and reverse orientation.
We demonstrated the expected sublinear scaling property for the gPBWT by building an index for chr22 with increasing numbers of samples.
Constructing the gPBWT for large haplotype sets proved difficult when using our particular implementation.
Progressive construction of the gPBWT in generic graphs was enabled by encoding the gPBWT into dynamic succinct data structures and adding a single haplotype thread and its inverse one at a time.
The overheads associated with the dynamic succinct data structures and the linear time required resulted in an infeasible serial construction algorithm for large haplotype sets, and all experiments were carried out using a partially ordered construction algorithm that worked using a VCF file as input.

The \emph{graph Burrows-Wheeler transform} (GBWT) \cite{siren2018haplotype} simplifies the data model used by gPBWT so that it is independent of {\tt vg} and can be applied to generic sequence graphs.
The most generic conceptual model for the GBWT is as the FM-index of a transformation of the graph's paths $p_1\ldots p_m \cup \overline{p_1}\ldots \overline{p_m}$ into the text $T = \$( p1 = n_i\ldots n_j) \ldots \$ (\overline{p_m} = \overline{n_j} \ldots \overline{n_i}) $ wherin the paths are rewritten as a series of characters representing node traversals in a large alphabet and delimited by a marker $\$$.
Unfortunately, this approach is technically challenging to implement due to the extremely large size of $T$ for moderately-sized haplotype sets embedded in variation graphs ($|T| \approx 10^{12}$ for the 1000GP  \cite{siren2018haplotype}).
Serializing the path set during construction is not feasible, which suggests a dynamic version of the model is required.
Also this approach it does not benefit from a number of features which distinguish paths in typical variation graphs.
Sequence graphs based on homology tend to be manifoldly partially ordered, as unique sequence will easily assemble into long coherently-ordered segments.
If we sort the identifier space of these graphs, then most haplotypes passing through the graph will be comprised of increasing or decreasing sequences.
This is even more true for VGs built from VCFs, although even there the issue of indels must be handled in a way that is very different from the assumptions that simplify the PBWT.
The nodes in these regions of the graph will also tend to have low degree.
These features suggest that a local compression of each's node's context in the graph can be achieved very efficiently.
The GBWT can thus benefit from a formulation in similar positional terms as those used in PBWT, with the augmentation of the positional model to record its local context in the graph.
The isolation of each node's context can also allow the construction of the GBWT to be completed in parallel over isolated segments of the variation graph, such as separate chromosomes, and these may be merged easily.

In the GBWT we break the full FM-index into per-node records, each of which encodes a header consisting of a description of the local edge transition alphabet $\Sigma_{n_i}$ through which the graph topology is maintained, and a body $BWT_{n_i}$ which is the subset of the full BWT specific to paths traversing $n_i$.
The GBWT supports essential FM-index operations including: $find(X) \Rightarrow [sp, ep]$ yielding the lexicographic range of suffixes starting with pattern $X$; $locate(sp, ep) \Rightarrow$ paths occuring in $SA[sp, ep]$; and $extract(j) \Rightarrow p_j$ which returns the $j$th path in the graph.
By encoding all paths in both orientations, the GBWT can be treated as a kind of FMD-index for haplotypes, allowing bidirectional search.
In principle, this means that the GBWT turn supports MEM-based haplotype matching, which has potential uses in genotype imputation, phasing, association mapping, and other population genetic and evolutionary assays.

The GBWT representation reflects a number of assumptions that tend to hold for most DNA sequence graphs.
Nodes tend to have low degree, which means the local alphabet size $|\Sigma_i|$ is small and we can afford to decompress a local alphabet encoding efficiently.
Most nodes are not traversed more than once by each path, so the $BWT_{n_i}$ remains small and can be accessed and modified in bounded time.
Due to relatedness among individuals in many species, it is sensible to assume that haplotypes will be highly repetitive, which allows for efficient RLE encoding of $BWT_{n_i}$.
The graph is sorted, and its identifier space has been compacted, which allows us to store the same information for the entire range of node identifiers in bounded memory with respect to $|N|$.
The graph tends to be locally ordered in most cases, which decreases the complexity of recording the graph topology in our records.

%GBWT builds an efficient dynamic model 

%To resolve the issues with construction faced by gPBWT, GBWT 

A \emph{dynamic GBWT} implementation presents the node records through an index over the range of $[min(i : n_i \in N), max(i : n_i \in N)]$, for each linking to its header, body, incoming edges, and haplotype identifiers.
Construction employs this model in a manner similar to RopeBWT2 \cite{li2014fast}, where batches of paths are insert into the index in a single step following the BCR construction algorithm to build the BWT for large sequence sets \cite{bauer2013lightweight}.
This process includes the new paths in the dynamic GBWT by rebuilding each node record affected by the extension.
At the end of each step the dynamic GBWT remains valid.
When constructed, the GBWT may be encoded in a compacted but immutable form that uses less memory.
By breaking the construction process apart for each chromosome and finally merging the compressed GBWTs, it is possible to build the GBWT for the entire 1000GP haplotype set around 30 hours.
The resulting GBWT requires $\approx 15$GB, with around half allocated to the GBWT structure itself and half to haplotype identifiers.
Incredibly, this structure requires less than 0.1 bits per mapping in the stored paths.
This demonstrates that it has fully exploited haplotype sharing in the 1000GP set, and we can expect that this compression rate will only improve with the addition of more samples.

\subsection{Generic disk backed indexes}
%*0.5p 0.5h*
\label{sec:generic_disk_backed_indexes}
I began the development of {\tt vg} alone, starting with schemas for the data models, then building an index of the graph using the disk-backed key/value store RocksDB\footnote{\url{https://github.com/vgteam/vg/blob/master/src/index.hpp}}.
In principle this would allow working with graphs that were many times larger than main memory.
I transcribed the data model into namespaces and sorted arrays written into the key/value store.
As compressors of various types could be applied to the sorted arrays backing RocksDB, the memory uses of this approach were ultimately similar to those for the final indexing models that I present here.
This encoding included the graph, alignments, and 
However, performance was far worse and the initial version of the aligner based on these systems could not achieve meaningful results using reasonable amountsof time for large graphs.
Ultimately, this flexible database model has remained important for some pipelines, in particular as a technique to organize alignments against the graph.
The ability of a disk backed index to exceed main memory limits makes it useful for exactly this kind of workload.
Unfortunately, other workloads such as sequence queries were untenable for large genomes, with reliable performance only possible if the entire index of spaced $27$-mers were cached in RAM, requiring nearly 200G in the case of the 1000GP variation graph.
The sorted disk-backed array does have the useful property of allowing prefix queries of the $k$-mer set, but this can easily be attained with GCSA2.
Finally, on the networked storage available in my instutitional setting, the construction costs for disk-backed index models were usually much worse than those of the XG and GCSA2 models.

\subsection{Coverage index}
%*1p 1h*


\section{Sequence alignment to the graph}
%*0.5p 0.5h*

\subsection{POA and GSSW}
%*2p 2h*

\subsection{Banded global alignment}
%*0.5p 0.5h*

\subsection{X-drop DP}
%*0.5p 0.5h*

\subsection{MEM finding}
%*2p 2h*


\subsection{Collinear chaining}
%*0.5p 0.5h*

\subsection{Distance estimation}
%*1p 1h*

\subsection{Chunked alignment}
%*1p 1h*

\subsection{Alignment surjection}
%*1p 1h*


\subsection{Finding the alignment}
%*0.5p 0.5h*

\subsection{Mapping quality}
%*2p 2h*


\section{Visualization}
%*0.5p 0.5h*

\subsection{Hierarchical layout}
%*0.5p 0.5h*

\subsection{Force directed models}
%*1p 1.5h*

\subsection{Linear time visualization}
%*1p 2h*


\section{Graph mutating algorithms}
%*0.5p 0.5h*

\subsection{Edit}
%*1p 1h*

\subsection{Pruning}
%*2.5p 3h*
\subsubsection{$k$-mer complexity reduction}
\subsubsection{High degree filter}

\subsection{Graph sorting}
%*1p 1h*

\subsection{Graph simplification}
%*0.5p 0.5h*


\section{Graphs as basis spaces for sequence data}
%*2p 2h*

\subsection{Coverage maps}
%*1p 1h*

\subsection{Variant calling}
%*3p 2h*

\subsection{Bubble decompositions and likelihood spaces}
%*2p 4h*

\subsection{MEM matching to the bidirectional GBWT}
%*1p 3h*


\section{Contributions to related computational methods}

% in this chapter I've described things that I've done
% as a consequence of being in this work, here are things that I helped other people to build
% list papers, give references

% gpBWT
% GCSA2
% ...
